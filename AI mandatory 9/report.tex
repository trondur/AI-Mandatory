\section*{1)}

A point in $n$-dimensions has $n$ numbers that indicate where it is located. In one dimension a point has the form $(x)$, in two dimension $(x, y)$, in three $(x, y, z)$ and so on. A unit hypercube in $n$ dimension has corner points in all permutations of the coordinates $(x, y, z...n)$ where the valid domain is $\{0, 1\}$. 

The corner points for a 2-hypercube are therefore:
\begin{center}
$(0,0), (0,1), (1,0), (1,1)$
\end{center}
and the corner points for a 3-hypercube are:
\begin{center}
$(0,0,0), (0,0,1), (0,1,0), (0,1,1), (1,0,0), (1,0,1), (1,1,0), (1,1,1)$
\end{center}
The number of possible permutations for a point in $n$-dimensions where the valid domain is $\{0, 1\}$ is $2^n$. A hypercube in $n$ dimensions therefore has $2^n$ corner points as other hypercubes are just scaled unit hypercubes. 

\section*{2)}
An inequality splits a dimension into two halves. In a 2-dimensional system the inequality $y \geq 0$ splits the $y$ dimension into two halves, one with all negative $y$'s which are not solutions to the inequality and one with positive $y$'s and 0 which all solutions to the inequality. To create a hypercube one needs two parallel inequality for each dimension. A $n$-hypercube therefore needs $2n$ inequalities to be represented.
This can be seen when for example creating unit 2-hypercube:\\
\begin{center}
$(y \geq 0)$, 
$(y \leq 1)$, 
$(x \geq 0)$, 
$(x \leq 1)$
\end{center}
\noindent
Or a unit 3-hypercube:\\
\begin{center}
$(y \geq 0)$, 
$(y \leq 1)$, 
$(x \geq 0)$, 
$(x \leq 1)$, 
$(z \geq 0)$, 
$(z \leq 1)$
\end{center}
Or a unit 4-hypercube:\\
\begin{center}
$(y \geq 0)$,
$(y \leq 1)$,
$(x \geq 0)$,
$(x \leq 1)$,
$(z \geq 0)$,
$(z \leq 1)$,
$(w \geq 0)$,
$(w \leq 1)$
\end{center}
\section*{3)}
I don't really understand the question, so i will try to answer both my interpretations of them.

Let us consider two corner points in a $n$-hypercube $A=(a_1, a_2, ..., a_n)$ and $B=(b_1, b_2, ..., b_n)$

If the two corners are adjacent, only one pivot is needed to reach to other corner, as increasing a slack variable to its maximum is equivalent to traveling along one of the inequalities from one corner point to another adjacent one.

If on the other hand the two corner points are as far as possible from each other, in other words $\sqrt{n}$, a minimum of $n$ pivots are needed to travel from $A$ to $B$. This can be observed when we consider a unit hypercube, where we need to travel at least once along every dimension to reach the corner point furthest away.

\section*{4)}
It \textit{is} possible to define an objective, maximize $\sum_{j=1}^{\infty} c_j x_j$, such that the simplex algorithm from a
feasible corner point of an n-dimensional hypercube uses more than n pivots to find an optimal solution. An example of this is when a degenerate problem creates a cycle, and the simplex algorithm does not terminate. It is however possible to implement cycle breaking in simplex, but it is often not done, as cyclic degenerate problems are rare. A degenerate example:\footnote{The example is taken from the slides from lecture 12}\\

\noindent
$x_5 = -0.5x_1 + 5.5x_2 + 2.5x_3 - 9x_4$\\ 
$x_6 = -0.5x_1 + 1.5x_2 + 0.5x_3 - x_4$\\
$x_7 = 1 - x_1$\\
$z = 10x_1 - 57x_2 - 9x_3 - 24x_4$


