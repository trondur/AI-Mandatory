\section*{1)}

A point in $n$-dimensions has $n$ numbers that indicate where it is located. In one dimension a point has the form $(x)$, in two dimension $(x, y)$, in three $(x, y, z)$ and so on. A unit hypercube in $n$ dimension has corner points in all permutations of the coordinates $(x, y, z...n)$ where the valid domain is $\{0, 1\}$. 

The corner points for a 2-hypercube are therefore:
\begin{center}
$(0,0), (0,1), (1,0), (1,1)$
\end{center}
and the corner points for a 3-hypercube are:
\begin{center}
$(0,0,0), (0,0,1), (0,1,0), (0,1,1), (1,0,0), (1,0,1), (1,1,0), (1,1,1)$
\end{center}
The number of possible permutations for a point in $n$-dimensions where the valid domain is $\{0, 1\}$ is $2^n$. A hypercube in $n$ dimensions therefore has $2^n$ corner points as other hypercubes are just a scaled unit hypercube. 

\section*{2)}
Wikipedia defines a hypercube as: "a closed, compact, convex figure whose 1-skeleton consists of groups of opposite parallel line segments aligned in each of the space's dimensions, perpendicular to each other and of the same length\footnote{https://en.wikipedia.org/wiki/Hypercube}"

Therefore a 2-hypercube, a hypercube in two dimensions, has two pairs of parallel lines.

Example of line coordinates in a 2-hypercube where $C_x$ is a constant and $x$ and $y$ are variables:\\ 
$(C_1, y)$ \\
$(C_1 +1, y)$ \\
$(x, C_2)$ \\
$(x+1, C_2)$ \\

\noindent Example of line coordinates in a 3-hypercube where $C_x$ is a constant and $x$, $y$ and $z$ are variables:\\

\begin{enumerate}
\itemsep-1.5em 
\item $(C_1, C_2, z)$ \\
\item $(C_1+1, C_2, z)$ \\
\item $(C_1, C_2+1, z)$ \\
\item $(C_1+1, C_2+1, z)$ \\
\item $(C_1, y, C_3)$ \\
\item $(C_1+1, y, C_3)$ \\
\item $(C_1, y, C_3+1)$ \\
\item $(C_1+1, y, C_3+1)$ \\
\item $(x, C_2, C_3)$ \\
\item $(x, C_2+1, C_3)$ \\
\item $(x, C_2, C_3+1)$ \\
\item $(x, C_2+1, C_3+1)$ \\
\end{enumerate}

\section*{3)}

\section*{4)}
